\chapter{Deploying Swifty}
\label{ch:deploy}

For fast deploying of Swifty components the \urlref{https://www.ansible.com}{Ansible}
recipes are provided. To be more detailed three virtual machines are provided:
\emph{crvm3}, \emph{crvm4} and \emph{crvm5} with \code{root:1q2w3e} as login and
password for administrator.

Make sure they are present on Ansible master node
(which we refer as \emph{ansible} next) \code{/etc/hostst}

\begin{lstlisting}
10.94.96.220            crvm3
10.94.96.219            crvm4
10.94.96.221            crvm5
\end{lstlisting}

Recipes are for Fedora 26 based system.

\section{Preparing nodes for deployment}
\label{sec:deploy-prep}

All nodes should be accessed from Ansible master via ssh keys, thus generate
key pairs and copy a public key to destination nodes. For example we refer
\code{crvmx-rsa} as a private key and \code{crvmx-rsa.pub} as a public one.

Copy public key to the destination machines from \emph{ansible} machine.

\begin{lstlisting}
ssh-copy-id -i ~/.ssh/crvmx-rsa.pub root@crvm3
ssh-copy-id -i ~/.ssh/crvmx-rsa.pub root@crvm4
ssh-copy-id -i ~/.ssh/crvmx-rsa.pub root@crvm5
\end{lstlisting}

Strictly speaking Ansible can operate with ssh passwords but using
key based access a way more secure.

Be default with minimal Fedora 26 setup the python2-dnf module is not installed
so probably one have to run

\begin{lstlisting}
ssh root@crvm3 dnf -y install python2-dnf
ssh root@crvm4 dnf -y install python2-dnf
ssh root@crvm5 dnf -y install python2-dnf
\end{lstlisting}

from the \emph{ansible} (depending if Ansible based on Python 2 or 3).

In \code{/etc/ansible/hosts} write

\begin{lstlisting}
[all:vars]
ansible_sudo_pass=1q2w3e

[swy-common]
crvm3
crvm4
crvm5

[swy-kube-master]
crvm3

[swy-kube-node]
crvm4
crvm5

[swy-mware]
crvm4

[swy-s3-master]
crvm3

[swy-s3-node]
crvm4
crvm5
\end{lstlisting}

Ping the machines

\begin{lstlisting}
[root@crvm3 ~]# ansible-3 swy-common -m ping
crvm4 | SUCCESS => {
    "changed": false,
    "failed": false,
    "ping": "pong"
}
crvm3 | SUCCESS => {
    "changed": false,
    "failed": false,
    "ping": "pong"
}
crvm5 | SUCCESS => {
    "changed": false,
    "failed": false,
    "ping": "pong"
}
\end{lstlisting}

Do not pay attention on crvm3 here we use it as \emph{ansible} node.

First we need to install general components on all \emph{crvmX} node.
For this sake run \code{mkdir -p /etc/ansible/playbooks}on the \emph{ansible}
node and put \code{swy-common.yaml} there with the following contents

\begin{lstlisting}
---
- hosts: '{{ nodes }}'
  remote_user: root
  tasks:
  - name: install rsync
    dnf: name=rsync state=present
  - name: install mc
    dnf: name=mc state=present
  - name: install vim
    dnf: name=vim-enhanced state=present
  - name: install libselinux-python
    dnf: name=libselinux-python state=present
  - name: disable firewall
    service: name=firewalld state=stopped enabled=no
  - name: disable selinux
    selinux: state=disabled
  - name: install nfs
    dnf: name=nfs-utils state=present
  - name: enable nfs
    service: name=nfs state=started enabled=yes
\end{lstlisting}

Then just run

\begin{lstlisting}
ansible-playbook /etc/ansible/playbooks/swy_common.yaml --extra-vars="nodes=swy_common"
\end{lstlisting}

which should install all general tools. While \emph{ansible} can reboot them
via recipe better do it manually for a while.

% Didn't debug servers reboot
%    #  - name: restart the machine
%    #    command: /sbin/shutdown -r +1
%    #    async: 0
%    #    poll: 0
%    #    ignore_errors: true
